In conclusion, we are happy with the result as we have managed to solve the problem of measuring students time
put into subjects for our customer.
We managed to test our application out in two subjects and collect a reasonable amount of data.

It is very likely that the application will be used in full in the next semester in iti0102 course and hopefully also in
iti0202, iti0201 and other similar programming subjects.

An avenue for future development is definitely integrating GTM to Charon/Moodle and tester as it allows more convenient usage for lecturers.
As linking the data with test results gives an opportunity to do some machine learning we hope someone can do it for their final thesis in the future.

As we are planning to continue using the app personally we are also planning to put more work into it in the summer to
also improve the GTM web application usage for personal projects.

For the web UI the main ideas for improvements are:
\begin{itemize}
    \item Add separate view for viewing user personal data over all repositories.
    \item Add branch/user-based filtering to the \textit{"Comparison"} tab and also add some statistics to it in numeric format.
    \item Allow students to view their ranking in Leaderboard.
    \item Allow users to create their own groups for better organization.
    \item Improve charts performance as there is some lag for longer periods of time.
    \item Move install instructions to our UI.
\end{itemize}

