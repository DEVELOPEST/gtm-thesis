\section{Groups system}\label{sec:group-system}

One of the requirements for our app was that it should be possible to group repositories together.
The grouping is needed to view statistics for multiple repositories at once and also compare groups of repositories.
The grouping system should be also capable of controlling user access rights to different repositories.

Requirements for grouping system
\begin{itemize}
    \item It shall be possible to group both repositories and groups already consisting of repositories.
    \item Single repository may be possible of multiple groups.
    \item Group access can be limited to only viewing summary (group total).
    \item Granting user an access group subgroups and then removing it shall have no side effects. (Accesses to subgroups shall not be persisted)
\end{itemize}

As the groups may consist of both groups and repositories we decided to automatically create group for every repository.
This means now groups can only consist of zero or more other groups.
If we need te get the repository we can fetch all the group ids that are accessible and then query from repositories database by
comparing repositories group id against previously fetched ids;
To fetch group with all of its subgroups a recursive Structured Query Language (SQL) query was needed as the groups'
hierarchy formulates a tree like structure.
Our database provider PostgreSQL supports recursive queries so there were no technical problems with implementing it on SQL database.
A simplified version of our group system database tables can be seen of Figure~\ref{fig:group_system}

\begin{figure}[h]
    \includegraphics[width=\textwidth]{figures/group_system}
    \caption{Application groups system}
    \label{fig:group_system}
\end{figure}

This tree like groups' hierarchy allows us to easily give and take user access to any group.
If we want to change access from only parent group to also all subgroups access, we can simply toggle access\_level\_recursive
variable.
If we remove one particular group access, all other accesses remain in place, meaning that you some group was accessible
also via some other group access, you still have the access.

\section{Security}\label{sec:scurity}
Our application holds data, that shall not be visible to all client and therefore some kind of authentication and authorization methods are required.
For the data stored in Git notes we decided no extra security is required, as the time data isn't more sensitive than the actual code.
The security of source code stored in git repository is handled by a client himself and Git providers.
If they wish to have some more protection for git notes, they can configure it themselves.
We only provide the option to have notes only stored locally (not pushed to origin).

For the web application we need to implement our own security measures.
For most basic usage we have simple username and password authentication.
Accounts that only have password authentication are not authorized to access any groups nor repositories unless Admin user
explicitly gives them access to any.

To automatically get access to repositories you are contributor of you need to authenticate yourself via OAuth2.
