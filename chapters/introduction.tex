The time spent on a programming task can be divided into two parts: time spent on writing code and supportive activities.
As the time spent on writing actual code directly contributes towards a solution, but time spent on supportive activities does not,
many companies seek to analyse this kind of data.

For working people, time spent on both writing code and all supportive activities will add up to total working hours.
As the total consists of only two components, if we manage to measure one we can easily calculate the other.
Measuring time on supportive activities is rather hard as they vary a lot in their nature.
Different time management tools can be used, but they usually rely on manual user input making them impractical.
Therefore, it is easiest to measure time spent on writing code.

Similarly, students time put into completing practical programming subject at university can be divided into three: time spent in lectures/labs,
time spent on doing research on your own and time spent on writing actual programs/solutions.
Time spent in lectures and labs can be easily measured as they have strict time schedules.
Similarly to supportive activities at work, measuring time spent on research has to rely on manual user input as research can be done in various ways some not involving computers at all.
Once again, an application for measuring time spent on writing code is needed.

As both companies and universities measure developers performance by measuring the amount tasks completed,
or the amount of code written it is also important to measure developer's efficiency in writing code.
Measuring statistics such as problems solved is hard as problems vary a lot, and their difficulty is complex to measure.
To simplify measuring developer efficiency we can look at the amount of code written as more code written usually results in more work done.

The amount of code written is usually measured in lines added and lines removed as the number of newlines usually describes the amount
of logical statements or functions used.
To have access statistics such as lines added and lines removed, some kind of access to a version control system is needed.

The goal of this project is to develop an application capable of both measuring time spent on writing code and also
displaying it in a readable format to a user.
The application must work with a minimal amount of manual input as manual input both discourages people to use it and may produce
unrealistic results.
People simply may give input based on their feelings, rather than actions.


\section{Project scope}\label{sec:project-scope}
The scope of this project is to build an application that solves the problem of measuring the time spent on writing programs for TalTech.
Although the application is developed for TalTech it should be possible to set it up for tracking time on personal or work projects.

The scope of this project consists of three main parts.
The goal of the first phase is to have a working app to track and save programmers time.
The second phase focuses on collecting the data together and storing it in a more accessible format.
The third phase is to visualize collected data to both lecturers and students.

As most TalTech subjects use Git as a version control system, the application developed must be able to work with
projects that use Git as their version control system.

It was discussed with the client that research needs to be done on already existing solutions for the problem and if possible,
the project should extend the already existing solution rather than building a new one.
It was also discussed that at least parts of the application that are installed on a client machine should be made open source.
