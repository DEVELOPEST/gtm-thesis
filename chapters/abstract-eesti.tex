Käesoleva bakalaureusetöö raames on tehtud Tallinna Tehnikaülikooli jaoks vabavaraline programmeerimise aja jälgimise ja visualiseerimise süsteem.
Eesmärk on anda nii õppejõule kui ka tudengitele ülevaade, millistele ülesannetele kõige rohkem aega kulub ning
kuidas on ajaline panus jaotunud.

Süsteem baseerub rakendusel \href{https://github.com/git-time-metric/gtm}{Git-Time-Metric}, mille ümber
ehitati üles ülejäänud süsteem.
Nüüd koosneb süsteem viiest erinevast rakendusest mis on jaotatud ka viite erinevasse kihti.
Kaks nendes paiknevad kliendi masinas ning ülejäänud kolm serveris.
Rakendustes on kasutatud Rust, Go, TypeScript ning Kotlin programmeerimise keeli, PostgreSQL andmebaasi.

Töö tulemusena täiendati \href{https://github.com/git-time-metric/gtm}{Git-Time-Metric} rakendust,
loodi Jetbrains'i rakendustega ühildumiseks \textit{plugin},
loodi \textit{Debian package}'sse pakendatud rakendus andmete lugemiseks paljudest Git repositooriumitest ning
valmistati \textit{backend} ja \textit{frontend} rakendused andmete mugaval kujul hoidmiseks ning visualiseerimiseks.

Valminud rakenduse veebiliides on kättesaadav aadressil \href{https://cs.ttu.ee/services/gtm/front}{https://cs.ttu.ee/services/gtm/front}
ning ülejäänud osad on leitavad GitHub'is \href{https://github.com/DEVELOPEST}{https://github.com/DEVELOPEST}.

Lõputöö on kirjutatud inglise keeles ning sisaldab teksti X leheküljel, Y peatükki, Z joonist, N tabelit.