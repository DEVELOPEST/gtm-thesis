Prior to the actual development, the general project framework was set up:
\begin{itemize}
    \item The roles and responsibilities were split.
    \item The rules and agreements on developing methodology were made.
    \item Environments were set up
\end{itemize}

It was agreed that Tavo Annus is responsible for DevOps and client side software.
Marten Jürgenson is responsible for a backend and frontend development.
As the client side software has less work to do, Tavo will be also working on both backend and
frontend on the later half of the project.

For the project management, GitHub was chosen as it is more common for users to browse project on GitHub rather than Gitlab.
Universities personal Gitlab server was ruled out as it does not allow non-university students to easily access open source parts of our application.

For communication channel, Discord was used for both text and sound/video communication.
The main reason was that we were both used to using Discord, it is free, and it allows reasonably well formatted code pastes / pinning.

Meetings with TalTech mentor were organized weekly on Teams to have continuous feedback on project.


\section{Use cases}\label{sec:use-cases}
To have more clear understanding of what are the problems for our potential users and how we can solve them three main use cases were discussed.
These use cases are:
\begin{enumerate}
    \item Time tracking for universities - measuring students work on programming subjects.
    \item Time tracking for companies - using time tracking at work to maximise developers efficiency.
    \item Time tracking for personal usage.
\end{enumerate}

For all of the use cases main benefits and concerns ere discussed.

\subsection{Time tracking for universities}\label{subsec:time-tracking-for-universities}
For universities the needs for time tracking are:
\begin{itemize}
    \item It has to be possible to have an overview of numerous students.
    \item The time tracking shall not be easily tricked to display bigger / smaller numbers.
    \item Application has to work on all commonly used operating systems (Windows 10, macOS and Linux).
    \item Application has to be easy to use and work with minimal user input.
    \item Application shall have clear purpose and access rights.
\end{itemize}

Possibility to have an overview of a group of students is needed to get feedback on subjects.
The only sensible way to have an overview of a volume of study in subject is to measure it on multiple students.
This also leads us to a requirement, that it shall not be easy for students to distort their time data.
Students commonly like to magnify their effort put into subjects and if it is easy to do, we have no idea if the data is distorted or not.
As there is a big number of students in some subjects and students have multiple subjects to attend it has to be easy to install the app.
There simply is no time to waste on helping all the students with more complex manual installation.
As universities have no right to enforce usage of some operating system, the app also has to work on all commonly used platforms.
Lastly it has to be clear for students, what the application is doing.
Otherwise, they may be afraid of invasion of privacy and not install the application.

\subsection{Time tracking for companies}\label{subsec:time-tracking-for-companies}
For companies the needs for time tracking are:
\begin{itemize}
    \item It needs to be clear, that no source code is leaked outside the company.
    \item The time tracking application has to work with repositories containing 10s thousands of commits.
    \item Time tracking data has to be persisted through complex Git operations such as rebasing.
    \item It has to be possible to filter time by issue.
    \item It has to be possible to filter time data in single repository by user.
    \item It shall be possible to dynamically group different repositories.
\end{itemize}

For companies more security concerns arise.
It is very important to protect source code, therefore they highly prefer the code to never leave their own hardware/servers.
As companies have numerous developers working on same project simultaneously for a long periods of time the application has to
remain working efficiently when used in repositories containing a large number of commits.
Furthermore, as there are many people working on same repository the time data has to persisted through all Git operations.
Otherwise, it will start limiting developers, and they will not use it.
To have clear overview of both work done by user and work done on certain issue it has to be possible to join time data over
multiple repositories and filter it by users and issues.

\subsection{Time tracking for personal use}\label{subsec:time-tracking-for-personal-use}
For personal usage, the main requirements for time tracking are:
\begin{itemize}
    \item Application has to work on all commonly used operating systems.
    \item Application has to be easy to install / use.
    \item Application has to be low cost, preferably free.
    \item Application needs to work with commonly used Git servers (Github, Gitlab.com)
\end{itemize}

For personal usage, similarly to universities the application has to work on all commonly used operating systems as otherwise we
would by greatly limiting the amount of potential users.
As there is usually no-one to help, the application installation has to be simple enough people with basic computer usage skills can do it.
One of the most important requirements for personal use is that the application has to be either very low cost or completely free.
Since people work on their personal projects only as much as they like, and they do it inconsistently they are not interested in paying for an app they might not even use.
As different developers have personal preferences for which Git server to use, we have to support all the most common ones.
As of beginning of 2021 they are Github.com anf Gitlab.com.


\section{Existing solutions}\label{sec:existing-solutions}
Before application design, some research was done to find already existing solutions.
The purpose of this task was to find out, if we could partially use some other programs and also get better understanding how
this kind of application could be built.


\subsection{Wakatime}\label{subsec:wakatime}
\href{https://wakatime.com/about}{Wakatime} is one of the most popular applications used for tracking your work.
It supports variety of Integrated Development Environments (IDEs) including JetBrains IDEs required by TalTech.
It has both it's IDE plugins and Command Line Interface (CLI) app licenced under Berkeley Software Distribution 3 Clause Licence (BSD-3).
It also has a public API that allows fetching time data from their servers.
The downside of using Wakatime is that it costs to see data older than 2 weeks and as it only temporarily
stores data on client machine, there is no easy way to access data without modifying already existent CLI app.
Wakatime also does not make use of Git statistics such as lines added and lines removed.
It only reads branch name, commit message and author data.
The reason behind this is likely it's architecture - it continuously sends heartbeats with required data to its backend.
If no internet connection exists, it stores these heartbeats locally in rather cryptic format.
As a result there is not much data stored locally meaning that the backend has to bundle time data to commits which becomes a very complicated task
once users start rewriting history, especially given that Git is a distributed version control system.
In the end of 2020 Wakatime also launched its Git integration that allows to see time spent per commit, but for previously
described reasons it gets confused after history rewrites and maps it based on timestamp.
This means that for example stashed time is added to wrong commit.
Wakatime's Git integration also does not support statistics such a lines added and deleted.

Although Wakatime's front and backend are private there are some open source alternatives.
Two most advanced alternatives are \href{https://github.com/mujx/hakatime}{Hakatime} licenced under Unlicense and
\href{https://github.com/muety/wakapi}{Wakapi} licenced under GPL-v3.
Both of them also have some kind of frontend capable of displaying data similarly to Wakatime's front end main page and
Hakatime has even some more graphs.
However, neither of them support any groups' system, and they don't have any Git integration.

As both of them have received numerous heavy updates since autumn 2020, they would be very strong candidates to base
our system on their already existing functionality, but that wasn't the case when we started with our project.


\subsection{Git-Time-Metric}\label{subsec:git-time-metric}
\href{https://github.com/git-time-metric/gtm}{Git-Time-Metric} is a popular open source time tracking plugin licenced under MIT licence.
It has plugins for most commonly used IDEs including JetBrains IDEs, VSCode, Vim and more.
It is closely tied to Git as it stores its time tracking data in Git notes.
This allows it to access more Git data such as lines added and lines removed for each commit.
Git-Time-Metric does not have a web interface (there are some half done projects meant to run only locally).
On the other hand, it has a very powerful and pretty CLI also capable of displaying data in a human-readable format.

One of the biggest downsides of Git-Time-Metric is that its active development has stopped in August 2019 and also its plugins haven't received many updates.
Since JetBrains IDEs have changed a lot since that time it doesn't work on newest IDE versions making it tricky to use.

The biggest design difference between~\nameref{subsec:wakatime} and Git-Time-Metric is that Git-Time-Metric stores all it's data locally.
This makes the Git repository own its time tracking data and user not to worry about external companies policies.
The CLI app is free and as long as there is a free Git server, you don't have to pay for anyone to persist you time data.


\subsection{DarkyenusTimeTracker}\label{subsec:darkyenus-time-tracker}
\href{https://github.com/Darkyenus/DarkyenusTimeTracker}{DarkyenusTimeTracker} is also an open source time tracker licenced under The Unlicensed.
It is only a plugin built for JetBrains IDEs, but it does support automatic recording.
It only stores time since last commit, but it does support automatic time logging to commit messages meaning that this data can be possibly collected by Jira or some other software.
Sadly Gitlab, nor Github currently support parsing time data from Git commit messages (Gitlab only support time logging directly under issue).
The other downside is that it only logs the total amount of time spent, no per file statistics.

Similarly to
\nameref{subsec:git-time-metric}, it does not have a web interface, and it would be very hard to make one as the time date is not clearly separated from Git commit messages.
It also doesn't have a CLI interface due to similar reasons as web app.


\subsection{General time tracking applications}\label{subsec:general-time-tracking-alternatives}
There are also several more general time tracking applications, some of which are
\href{https://toggl.com/}{Toggl}, \href{https://www.forestapp.cc/}{Forest}, \href{https://www.rescuetime.com/}{RescueTime} and
\href{https://clockify.me/}{Clockify}.
Although they are designed for more general use, it is important to also review their main features to see,
if any of them can be extended to be suitable for our use case.
\cite{general-time-tracking-alternatives}

These applications are available for multiple platforms and devices, including phones.
Most of them have plugins for different applications, including IDE-s in some cases, but they always require some manual
input for starting tracking.
Usually the manual input is required to choose the task and rest of the tracking is automatic.
As these apps are not build for programmers, they lack Git related statistics such as lines added and removed.
The most accurate connection, between time and code written is usually the task/branch name.

In short, these applications solve different problem of measuring work time, rather than the problem of measuring
code writing statistics.

\subsection{Existing plugin analysis results}\label{subsec:existing-plugin-analysis-results}
As a result of analysis, a table was draw to visualize differences between existing solutions.
Toggl was chosen to represent general time tracking applications, as it has the most programming related features
out of the ones described above.
Features comparison is visible in Table~\ref{tab:comparison-results}.

\begin{table}[h]
    \centering
    \begin{tabular}{ | p{4cm} | p{2.25cm} | p{2.25cm} | p{2.25cm} | p{2.25cm} | }
        \hline
         & Wakatime & Git-Time-Metric & Darkyenus-TimeTracker & Toggl\\
        \hline
        Supports most common computer operating systems & Yes & Yes & Yes & Yes\\
        \hline
        Is fully automatic & Yes & Yes & Yes & No\\
        \hline
        Has Git integration & Only in paid version & Yes & No & Only with issues\\
        \hline
        Supports tracking non programming activities & Yes & No & No & Yes\\
        \hline
        Is open source & Partially & Yes & Yes & No\\
        \hline
        Can be used locally & No & Yes & Yes & No\\
        \hline
    \end{tabular}
    \caption{Existing solutions features comparison.}
    \label{tab:comparison-results}
\end{table}

During the analysis there were two main ideas which way to go:
\begin{enumerate}
    \item Build a Git statistics support for~\nameref{subsec:wakatime}.
    \item Add some features to~\nameref{subsec:git-time-metric} and rebuild (some of) its plugins.
\end{enumerate}

Toggl was not viable for us as it is built to solve different kind of problem.
\nameref{subsec:darkyenus-time-tracker} was ruled out as it lacked features useful for us.
Non the less, it was decided that we can use it as a base for building JetBrains plugin.

\nameref{subsec:wakatime} seems to be the best choice for targeting only companies.
It is already built for companies to use and it has been tested a lot.
It's also more delicate than~\nameref{subsec:git-time-metric} when handling source code as it supports obfuscating file names and paths.

On the other hand, building a Git statistics support for~\nameref{subsec:wakatime} also requires us to build our own backend
for all the data or pay to access~\nameref{subsec:wakatime} API. % TODO: See oli nii 2020 sugisel, kas peaks mainima et nuud on alternatiivid paremad?
Either way an external database is needed to store all its Git related statistics.
Whilst it has great benefits, it is very hard to implement as it would require adding extra features to all of its plugins.
As Git history rewrites may cause conflicts it also needs either a complex system to solve these conflicts or just ignore history overwrites.
The second option would mean that whilst the time data saved would very closely reflect developer work in the short term it might get
confusing in the long run.
As a result of numerous rebases and squashes in different branches it becomes difficult to understand where the time data belongs.
The question arises: Why would we want to have time data attached to many commits if we don't really care about the commits and just squash them?
For us the answer was that we should also squash the time data to only "point" to single squashed commit, but it is too complex to achieve.

When adding Git statistics to~\nameref{subsec:wakatime} it is rather hard to persist its security features such as obfuscating file names.
Aa a result, whilst~\nameref{subsec:wakatime} is currently the most ready for commercial use, it is hard to implement extra features.

Adding features such as automatic notes fetching and pushing to~\nameref{subsec:git-time-metric} on the other hand is way simpler.
Whilst the rewriting of IDE plugins is a lot of work it's still less than building a system for~\nameref{subsec:wakatime} capable of handling Git rebases.
It is also important to mention thant~\nameref{subsec:git-time-metric} architecture makes it difficult to record indirectly programming related tasks time
such as searching for help online.
The reason for it is that it is very inconsistent to tie time spent outside editing Git repository files to certain Git repository.
In this project context we don't see this as a big problem as we are mainly targeting time spent inside IDE.
Trying to record internet browsing data or time spent on other programs may be seen as an invasion of privacy,
and it might greatly reduce the amount of users interested in our app.
