Prior to the actual development, the general project framework was sey up:
\begin{itemize}
    \item The roles and responsibilities were split.
    \item The rules and agreements on developing methodology were made.
    \item Environments were set up
\end{itemize}

It was agreed that Tavo Annus is responsible for DevOps and client side software.
Marten Jürgenson is responsible for a backend and frontend development.
As the client side software has less work to do, Tavo will be also working on both backend and
frontend on the later half of the project.

For the project management, GitHub was chosen as it is more common for users to browse project on GitHub rather than Gitlab.
Universities personal Gitlab server was ruled out as it does not allow non-university students to easily access open source parts of our application.

As a communication channel, Discord was used for both text and sound/video communication.
The main reason was that we were both used to using Discord, it is free, and it allows reasonably well formatted code pastes / pinning.

Meetings with TalTech mentor were organized weekly on Teams to have continuous feedback on project.

\section{Existing solutions}\label{sec:existing-solutions}
Before application design, some research was done to find already existing solutions.
The purpose of this task was to find out, if we could partially use some other programs and also get better understanding how
this kind of application could be built.


\subsection{Wakatime}\label{subsec:wakatime}
\href{https://wakatime.com/about}{Wakatime} is one of most popular applications used for tracking your work.
It supports variety of Integrated Development Environments (IDEs) including JetBrains IDEs required by TalTech.
It has both it's IDE plugins and Command Line Interface (CLI) app licenced under BSD-3 licence.
It also has a public API that allows fetching time data from their servers.
The downside of using Wakatime is that firstly it costs to see data older than 2 weeks and as it only temporarily
stores data on client machine, there is no easy way to access data without modifying already existent CLI app.
Wakatime also does not make use git statistics such as lines added and lines removed.
It only reads branch name, commit message and author data.
The reason behind this is likely it's architecture - it continuously sends heartbeats with required data to it's backed.
If no internet connection exists, it stores these heartbeats locally in rather cryptic format.
As a result there is not much data stored locally meaning that the backend has to bundle time data to commits which becomes a very complicated task
once users start rewriting history, especially given that Git is a distributed version control system.


\subsection{Git-Time-Metric}\label{subsec:git-time-metric}
\href{https://github.com/git-time-metric/gtm}{Git-Time-Metric} is a popular open source time tracking plugin licenced under MIT licence.
It has plugins for most commonly used IDEs including JetBrains IDEs, VSCode, Vim and more.
It is closely tied to git as it stores it's time tracking data in git notes.
This allows it to access more Git data such as lines added and lines removed for each commit.
Git-Time-Metric does not have a web interface (there are some half done projects meant to run only locally).
On the other hand, it has a very powerful and pretty CLI also capable of displaying data in a human-readable format.

One of the biggest downsides of Git-Time-Metric is that its active development has stopped in August 2019 and also it's plugins haven't received many updates.
Since JetBrains IDEs have changed a lot since that time it doesn't work on newest IDE versions making it tricky to use.

The biggest design difference between~\nameref{subsec:wakatime} and Git-Time-Metric is that Git-Time-Metric stores all it's data locally.
This makes the Git repository own it's time tracking data and not to worry about external companies policies.
There is always a CLI app capable of continuing to record and also display the data tracked.


\subsection{DarkyenusTimeTracker}\label{subsec:darkyenus-time-tracker}
\href{https://github.com/Darkyenus/DarkyenusTimeTracker}{DarkyenusTimeTracker} is also an open source time tracker licenced under The Unlicensed.
It is only a plugin built for JetBrains IDEs, but it does support automatic recording.
It only stores time since last commit, but it does support automatic time logging to commit messages meaning that this data can be possibly collected via Jira or some other software.
Sadly Gitlab, nor Github currently support parsing time data from git commit messages (Gitlab only support time logging directly under issue).
The other downside is that it only logs the total amount of time spent, no per file statistics.

Similarly to
\ref{subsec:git-time-metric}, it does not have a web interface, and it would be very hard to make one as the time date is not clearly separated from Git commit messages.
It also doesn't have a CLI interface due to similar reasons as web app.

\subsection{Existing plugin analysis results}\label{subsec:existing-plugin-analysis-results}
During the analysis there were two main ideas which way to go:
\begin{enumerate}
    \item Build a Git statistics support for~\nameref{subsec:wakatime}.
    \item Add some features to~\nameref{subsec:git-time-metric} and rebuild (some of) its plugins.
\end{enumerate}

\nameref{subsec:darkyenus-time-tracker} was ruled out as it lacked features useful for us.
Non the less, it was decided that we can use it as a base for building JetBrains plugin.

Building a Git statistics support for~\nameref{subsec:wakatime} also requires us to build our own backed for all the data or pay to access~\nameref{subsec:wakatime} API.
Either way an external database is needed to store all its Git related statistics.
Whilst it has great benefits, it is very hard to implement as it would require adding extra features to all most all it's plugins.
As Git history rewrites may cause conflicts it also needs either a complex system to solve these conflicts or just ignore history overwrites.
The second option would mean that whilst the time data saved would very closely reflect developer work in the short term it might get
confusing in the long run.
As a result of numerous rebases and squashes in different branches it becomes difficult to understand where the time data belongs.
The question arises: Why would we want to have time data attached to manny commits if don't really care about the commits and just squash them?
For us the answer was that we should also squash the time data to only "point" to single squashed commit, but it is too complex to achieve.

Adding features such as automatic notes fetching and pushing to~\nameref{subsec:git-time-metric} on the other hand is way simpler.
Whilst the rewriting of IDE plugins is a lot of work it's still less than building a system for~\nameref{subsec:wakatime} capable of handling git rebases.
% TODO: More discussion