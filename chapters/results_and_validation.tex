As a result of this project, collection of applications was built to allow automatic time tracking for Git projects.
\begin{itemize}
    \item Original~\nameref{subsec:git-time-metric} application was forked (named to gtm-core) and new features were added
    alongside numerous bug fixes.
    \item Jetbrains IDE plugin (gtm-jetbrains) was created to integrate gtm-core functionality into IDE and enable tracking without manual input.
    \item Gtm-sync client (gtm-sync) was built to safely fetch data with Git notes from Git, parse data from notes and sync up to backend.
    \item Web server backend (gtm-api) was built to persist data and serve it to frontend.
    \item Web application frontend (gtm-front) was built to visualize data in an easy-to-understand format.
\end{itemize}

To validate the success of the project it is important to measure, how many students are using application,
how satisfied the students are with the application and how satisfied the lecturers are with the application.

Measuring amount of users using the app is easiest, but the most important one.
If there is a small amount of users using the application then the application has failed its purpose and other
aspects can be only used to analyse how to improve.
Finding out the amount of users is rather easy, as the application itself stores this data and admin user can easily access it.
In iti0201 course we have time tracked from at least 34 different Git emails and in iti0202 we have data from at least 55 different git emails.
As some people may have different Git emails on different computers, or they may change email during course, we cannot say it
is the same amount of students but guessing by emails we can say it is roughly 50\% of students in these courses.
We are very happy with this percentage of students already using the app as the application, especially the parts
responsible for displaying data to users, were still in development throughout the course.

Although the amount of users logically should reflect on their satisfaction in application, it was still important to
get more feedback from users to see what parts of application need improvements.
To get more feedback from students an anonymous questionnaire was created and sent to all students participating in iti0201, iti0202
and iti0301 courses in spring semester of 2020/2021 study year.
In the questionnaire, there were questions to both our users, and students who didn't try the app out had stopped using it.

In total there were 18 responses, 13 were using GTM as of answering the survey, 4 had tried it but stopped using it and one respondent hadn't tried it.
The person, who had decided not to try out the app stated, that he didn't want to share his personal data with us and also brought out,
that he believes measuring only time spent writing code is biased as is gives incomplete picture about total time spent.

Out of the 4 users, who had tried the app but stopped using it, one said he stopped using the app as he has more convenient alternative.
Two respondents said they had some problems when using the app and didn't feel like continuing to use it.
One person said he didn't like the idea of sharing such private data as time data and stopped using the app due to that.
Out of these responses it seems, that it is important make the application usage more convenient to keep our users.
We already have tweaked our application over the semester for easier usage, but there definitely is a lot more room for improvements.

Out of the 13 people still using GTM 9 said they are using it in either iti0202 or iti0201, and 5 that they are using in iti0301.
These are the subjects that we and also lecturers endorsed students to use GTM at, so we were expecting most of our
users to use it in one of these subjects.
Three respondents also said they are using GTM in other university subjects and also 3 were using GTM in personal projects.
We are very happy that some people have been using the app also outside the subjects where lecturer endorses the usage of app
as it shows they like what the app is doing, and it solves some problem for them.

GTM users brought out that they like the precision of time and that it's tied with commits and files, so they can see
on what task the time was spent.
Users also liked the web UI and that it has different ways to visualize the data both on web and also on command line.
It was stated, that the app not requiring any manual input after initializing the tracking for first time is a big plus as well.

From the downsides, it was brought out that viewing time data from terminal is inconvenient for many users.
It was also stated that the installation could have been easier and there could be a better web UI.
Also, it was stated that the time widget in Jetbrains IDE-s sometimes stops working when multiple IDE-s are open
which can be considered a bug.

The issue of having to do some things from command line was actually solved in the later half of the semester, but it seems
we should have communicated it better as some people were not aware of it.
Only slightly more that 50\% of the users were using GTM web app with more than 30\% never having visited it.
Partially it might be caused by us not having the web UI ready for students when we first introduced the app to them.

% TODO: Ago ja Gert-i feedback
