As a result of this project, collection of applications was built to allow seamless time tracking for git projects.
\begin{itemize}
    \item Original~\nameref{subsec:git-time-metric} application was forked (named to gtm-core) and new features were added
    alongside numerous bug fixes.
    \item Jetbrains IDE plugin (gtm-jetbrains) was created to integrate gtm-core functionality into IDE and enable tracking without manual input.
    \item Gtm-sync client (gtm-sync) was built to safely fetch data from git and parse data from notes into JSON.
    \item Web server backend (gtm-api) was built to persist data and serve it to frontend.
    \item Web application front end (gtm-front) was built to visualize data in an easy-to-understand format.
\end{itemize}

To validate the success of the project it is important to measure, how many students are using application,
how satisfied the students are with the application and how satisfied the lecturers are with the application.

Measuring amount of users using the app is easiest, but the most important one as if there is a small amount of users using
the application then the application has failed its purpose and other aspects can be only used to analyze, how
to improve.
Finding out the amount of users is rather easy, as the application itself stores this data and admin user can easily access it.
% TODO: Numbrid ja j2reldused

Although the amount of users logically should reflect on their satisfaction in application, it was still important to
get more feedback from users to see what parts of application need improvement.
To get more feedback from students a Google form was created and sent to all students participating in iti0201, iti0202
and iti0301 courses in spring semester of 2020/2021 study year.
% TODO: Küsimused ja analüüs

% TODO: Ago ja Gert-i feedback
